\clearpage
\vspace{1cm}
\section{Conclusions} \label{sec:conclusions}
\vspace{1cm}

The main objective of this thesis was to perform a phenomenological analysis of the production of $Z^{\prime}$ bosons at the LHC. In order to do this, a review of some important aspects of the SM and its theory were explained in Section \ref{sec:teo}. In Section \ref{sec:Zprime}, a motivation for beyond the SM theories, together with a brief review of current searches of $Z^{\prime}$ bosons in different models, and some basic aspects of the theory of this new vector boson were made. Section \ref{sec:colliders} explains some of the main variables used in particle physics, together with an explanation of the CMS experiment at the LHC and its experimental constraints, necessary for the determination of the selection criteria required to ensure events meet the specifications of the detector. Also, a brief explanation of the different software used for the analysis, together with the workflow of the code developed for the analysis was incorporated. Finally, in Section \ref{sec:results}, the different selection criteria was applied to the different signals generated, in order to perform a statistical analysis of the expected sensitivity of searches of the $Z^{\prime}$ boson at the CMS.

For this thesis, the signals and relevant backgrounds were studied at MadGraph level for high and low $Z^{\prime}$ masses, without the use of software such as Delphes or Pythia to generate shower and detector effects. At this level, different preliminary selection criteria were applied to the events, exploiting the similar topology of the signals and backgrounds, characteristic of $t\overline t$ processes, in order to differentiate measurable processes of interest from other possible processes taking place at the LHC. Further selection criteria was established to maximize the signal significance in the high $Z^{\prime}$ mass limit and showed optimal results for eliminating background data. With the results obtained, a preliminary statistical analysis was carried out, obtaining significances for the different signals ranging from 1.659 up to a value of 3.967. Even though significances of values around 5 were not obtained, the values presented a significant prevalence of the signals over the backgrounds, so it is expected to be able to achieve higher significances values with a higher luminosity, which could result in a potential discovery at the LHC.

As further work, it is proposed the possibility of using multi-varied analysis techniques for the fully-hadronic channel. Furthermore, it is still required the determination of efficient selection criteria to maximize the signal significance for the low $Z^{\prime}$ mass limit, which represents a complicated experimental region for searches of $Z^{\prime}$ bosons. Also, an accurate method for tagging the muons coming from the $t\overline t$ annihilation in the muon channel is required, since in the low $Z^{\prime}$ signals this proved to be a not so trivial task, making for considerably smaller significances in the muon channel. Finally, to further improve the results of this work for a LHC production study, a code that analyses samples taking into account detector and shower effects is currently under development.