\clearpage
\vspace{1cm}
\section{Introduction} \label{sec:intro}
\vspace{1cm}

\subsection{Particle Physics} \label{ssec:partphys}

The origin of particle physics can be roughly traced back to 1897, year in which J. J. Thompson discovered the electron, correctly proposing them as fundamental constituents of matter, unleashing a series of studies of the composition of atoms, until 1932, with Chadwick's discovery of the neutron. This discovery was very important, as it helped to understand the discrepancy between masses and the amount of electrons in atoms (as these are neutral in charge), establishing that atoms were made up of protons, electrons and neutrons. But this was just the beginning of particle physics, as an enormous amount of efforts and theories to try to explain and discover the fundamental building blocks of the universe was already in motion \cite{Griffiths}.

The product of these series of attempts, over almost a century, gave birth to what today is known as the Standard Model (SM) of particles. Figure \ref{SM} shows a sketch that summarizes the known funtamental particles in th SM. The SM is a theory that describes the fundamental particles in nature, this is, all the elementary building blocks that compose all matter in the universe, and the fundamental forces. The catch to the SM (along with some other problems) is that it actually encompasses three of the four fundamental forces of nature: electromagnetic, weak, and strong forces, noticeably not including gravity.

\begin{figure}[ht]
    \centering
    \includegraphics[width=10cm]{Sections/images/Imagenes/SM.png}
    \vspace{-1\baselineskip}
    \caption{Standard model of particles, where the antiparticles for the fermions are not included. Taken from \cite{Goldberg}.}
    \label{SM}
\end{figure}

The main reason for this, is that formulating a consistent theory that successfully includes gravity has proven to be more difficult than expected \cite{Griffiths}, and even though this is a time bomb, as the incorporation of gravity is a problem that will have to be addressed in the future (and is actually being dealt with right now), as can be seen in Table \ref{forces}, gravity is by far the weakest of all four fundamental forces, by at least around 29 orders of magnitude. This makes gravity only relevant at scales of the Planck energy (Figure \ref{Epl}), an energy level still way above the energy scales that are currently being studied, and from the energies at which particles predicted by the SM have been found, making it possible to consider gravity negligible at these energy scales.

\begin{table}[h!]
\centering
\caption{Table of the strength of the four different fundamental forces of nature. Adapted from \cite{Griffiths}.}
\label{forces}
\begin{tabular}{llll}
\hline
\hline
Force & Strength & Theory & Mediator \\
\hline
Strong & 10 & Chromodynamics & Gluon \\
Electromagnetic & $10^{-2}$ & Electrodynamics & Photon \\
Weak & $10^{-13}$ & Flavordynamics & $W$ and $Z$ \\
Gravitational & $10^{-42}$ & Geometrodynamics & Graviton \\
\hline
\hline
\end{tabular}
\end{table}

\begin{figure}[ht]
    \centering
    \includegraphics[width=14cm]{Sections/images/Imagenes/Epl.png}
    \caption{Energy (up) and length (down) scales relevant in particle physics, where $E_{pl}$ is the Planck energy, and $m$ different particles masses in natural units. Taken from \cite{Goldberg}.}
    \label{Epl}
\end{figure}

As can be seen in the Figure \ref{SM}, the model is divided in two fundamental blocks, being these fermions and bosons. The reason why such division is made is explained below. It is important to note that this division is not merely due to pure statistical classification, based upon different ``rules'' (statistics, or commutation rules \cite{Dirac}) that fermions and bosons, follow, but rather a more fundamental underlaying reason.

Fermions: These are the fundamental particles that compose all matter, and are divided in two groups, quarks (q) and leptons (l). These two groups come in three different generations, represented as rows in Figure \ref{SM}, i.e for quarks the first generation is the up (u) and down (d) quarks, and for leptons the electron ($e^-$) and electron neutrino ($\nu_e$). The second generation is the charm (c) and strange (s) quarks, and the muon ($\mu^-$) and muon neutrino ($\nu_{\mu}$), and the third, top (t) and bottom (b) quarks, and tau ($\tau^-$), tau neutrino ($\nu_{\tau}$). 

The difference among these families is their mass, as they become heavier than the previous family, but their charges and spin are the same \cite{Langacker}. Each of these particles comes with its own antiparticle, for the $e^-$, $\mu^-$ and $\tau^-$ we denote them with a + sign (for example, the antiparticle of the electron, the positron, is denoted as $e^+$), as they have the opposite electric charge. For neutrinos, we denote antiparticles with an overline, for example the electron antineutrino is $\overline \nu_{e}$, as they do not have charge (this overline notation is also used for antiquarks). Due to their notorious lack of charge, the main characteristic of neutral antiparticles is their inverted helicity, which is the projection of the particle's spin onto its direction of movement, a measurable quantity through beta decay.

Heavier families are not stable and decay on lighter families. Therefore, stable matter is composed of quarks and leptons from the first family. Experimentally, it has been observed that matter and anti-matter are produced in equal amounts in the radiative decay processes. Nevertheless, the universe is made up of matter and not anti-matter, which is currently a puzzle in particle physics. This is commonly known and as the baryon asymmetry problem \cite{Langacker}. Particles that are made up from quarks are called hadrons, that are separated into two groups: baryons are particles made up at least of three quarks, such as the proton (p) and the neutron (n) (exotic baryons are made of five quarks, known as pentaquarks), and mesons are particles made up from a quark-antiquark pair, such as the pion ($\pi^0, \pi^{\pm}$) and the kaon ($k^0, k^{\pm}$) \cite{Griffiths}. For example, all elements in the periodic table are made up from baryons (p, n) and leptons ($e^-$), as mentioned before.

Bosons: Bosons are the force carriers of the different interactions. As the SM is written in a formalism called quantum field theory (QFT), it makes a change of paradigm, where the classical fields that generate the interactions, such as the electromagnetic fields, are now quantized, and their quanta, their respective particles, in this case the bosons, inspired in the ladder operator from quantum mechanics, are responsible of the interactions, which are but a perturbation of the fields \cite{Lahiri}. For electromagnetism, the mediator is the photon ($\gamma$), for the weak interaction the $W^{\pm}$ and $Z^0$ bosons, and for the strong force there are eight gluons ($g$) \cite{Goldberg}. 

Something important to note is that the number of bosons is not a coincidence, as the SM has three sectors (the electromagnetic, weak and strong), the elements of each of these sectors transform, maintaining the theory invariant, under $U(1), SU(2)$ and $SU(3)$ symmetries respectively. Moreover, the number of gauge bosons, surging from the gauge invariance each of these groups has, is given by the number of infinitesimal generators of the group, where the $U(1)$ group has one gauge boson, and the $SU(N)$ groups have $N^2-1$ generators, so $SU(2)$ has three gauge bosons and $SU(3)$ eight \cite{Goldberg}, the force carriers mentioned before.

In conclusion, the SM has a $U(1)_Y\times SU(2)_L\times SU(3)_C$ group symmetry \cite{Langacker}. The sub-index $Y$ is for hypercharge, a new quantum number used to define the electric charge. $L$ is left, which denotes that the weak sector has a $SU(2)$ symmetry only for left-hand chirality state particles \cite{Langacker}\cite{Nagashima} (denoted with a $L$ subscript), as the weak sector violates parity, due to the absence of right-handed neutrinos \cite{Hadjiivanov}. Because of this, the doublet $(e^-, \nu_e)_L$ transform under the $SU(2)$ symmetry, while the singlet $(e^-)_R$ transforms trivially. Finally, $C$ is for color, as particles of the $SU(3)$ sector (quarks and gluons) have color charge, which is responsible of making particles interact strongly. All particles composed of quarks are neutral in color charge, so only quarks themselves take part in the strong interaction. 

The last SM boson comes from the spontaneous symmetry breaking (SSB) of the electroweak sector $U(1)_Y\times SU(2)_L$ below certain energy level $E_{weak}$. Via this SSB, the electroweak sector turns into the electromagnetic sector $U(1)_{em}$, where particles only interact through the electromagnetic field \cite{Leike}. From this SSB arises the Higgs field, which permeates all space, interacting with particles through the Higgs boson, giving them mass, determined by how strongly they couple to this field \cite{Nagashima}.

\subsection{Methodology} \label{ssec:methodology}

Currently the SM is the most accurate theory we have to describe particles and their interactions at the most fundamental level, but it is not a complete theory, as it has some problems as will be discussed later. These problems are a clear indicator of the need for more physics, as improving the theoretical model is needed to correctly describe the phenomena observed experimentally. The main purpose of this thesis is to study the possibility of physics beyond the SM. In particular, we explore the hypothetical production of a $Z^{\prime}$ boson under the conditions of the Large Hadron Collider (LHC), via the fusion of a top and anti-top quark pair. Currently, the LHC is the most powerful particle collider in the world, built to study several open questions in fundamental particle physics.

As the name implies it, this boson is very similar to the $Z^0$ boson from the SM. This boson has an electric charge of 0, and a spin of 1, and even in certain models such as the Sequential Standard Model (SSM), it has the exact same couplings to fermions as the $Z^0$. Nevertheless, its mass is an open parameter in most models and can range from GeV to TeV scales, depending on the specific theory. Chapter \ref{sec:Zprime}) will describe in more detail generalities on the $Z^{\prime}$ physics.

The $Z^{\prime}$ is of great importance in the search of new physics, because as mentioned before, many theories predict the existence of this new particle, which could not only be one of the first indicators of possible new physics, but could also show the correct way to proceed to a new theory, as currently many different models that attempt to further explain phenomena not accounted for by the SM exist, with no experimental evidence capable of ruling them out.

In order to do this, a brief review of the theory composing the SM and some basic methods for computing conserved quantities, Hamiltonians, transition amplitudes, etc is made in Section \ref{sec:teo}. In Section \ref{sec:Zprime}, different theories and concepts concerning the $Z^{\prime}$ boson are mentioned. Then, Section \ref{sec:colliders} gives a short explanation on how detectors (in particular the CMS experiment at the LHC) work, also introducing some of the important variables used for phenomenological analyses of particle interactions, and a basic explanation of the different simulation software used for data generation and analysis. Finally in Section \ref{sec:results} the different cuts and selection criteria applied, together with the statistical analysis and results obtained are showed, and in Section \ref{sec:conclusions} the conclusions are given.

In order to accomplish this, the objectives for this thesis are:

\begin{itemize}
    \item Determine the dependence on the couplings and the mass of the $Z^{\prime}$ of the production cross section of the process of interest.
    \item Produce simulations for signals and background relevant for the study, including detector effects.
    \item Find the kinematic and topological variables that separate best the signal and backgrounds.
    \item Determine the optimal points to place thresholds that maximize signal significance on the variables of interest.
    \item Perform a preliminary statistical analysis to establish the expected sensitivity in the search at the CMS experiment in the LHC.
\end{itemize}