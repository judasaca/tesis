\clearpage
\vspace{1cm}
\section{Beyond the SM: $Z^{\prime}$} \label{sec:Zprime}
\vspace{1cm}

In the following chapter, a brief explanation of why new physics are needed to complete the SM is made (Section \ref{ssec:newphysics}), in which the importance of the $Z^{\prime}$ for searches of new physics in several different models is explained. In Section \ref{ssec:zpmodel}, the basic theoretical concepts and implications of the addition of a new heavy boson ($Z^{\prime}$) to the SM are introduced, and finally, in Section \ref{ssec:searcheszp}, results on the latest searches of the $Z^{\prime}$ boson conducted at the LHC are discussed, showing the actual upper limit in mass where the $Z^{\prime}$ as been excluded.

\subsection{The need of new physics} \label{ssec:newphysics}

Even after the enormous success of the SM, there are some questions that still elude us to this day, questions that the SM does not account for and cannot hope to explain. Because of this phenomena, there is a need of more physics, but in regard of the amazing experimental data coinciding with the predictions made by the SM, it is known that it is not necessary to get rid of the actual standard model, but rather to further improve it \cite{Griffiths}. 

Some of these problems the SM has are: the amount of free parameters, including masses of particles which are not predicted, charge quantization, which the SM does not give a justification as to why particles have charges multiple of $e/3$, parity violation, neutrinos oscillation, which implies that neutrinos are not massless, baryon asymmetry, the absence of promising candidates for dark matter, and many more \cite{Goldberg}\cite{Langacker}.

In order to try to solve these problems, and because of the complicated group structure of the SM, some theories have tried to further unify the interactions, in a way similar as how the electroweak symmetry ($SU(2)_L\times U(1)_Y$) breaks into electromagnetism ($U(1)_{em}$) below some energy scale $E<E_{weak}$. These theories are called grand unified theories (GUTs), and suggest that above some energy $E_{GUT}$, the strong and electroweak interactions can be described by a single group $G$, such that below this energy $E_{GUT}$, the group splits into the well known $SU(3)_C\times SU(2)_L \times U(1)_Y$ symmetry \cite{Leike}. Other type of theories that aim to unify the forces are string theories, which also try to include gravity, supersymmetry (SUSY) models \cite{Langacker}\cite{Martin}, such as supergravity and super string theories, in which a superpartner (or sparticle) for each particle of the SM is added, with a difference of 1/2 in the spin, making pairs of boson-fermion, and restoring the boson-fermion symmetry.

In later years, GUT models have been of increasing interest, where it was shown by H. Georgi and S. L. Glashow that the smallest simple gauge group $G$ that can contain the SM is $SU(5)$ \cite{Leike}. In this case there is no room for new neutral gauge bosons, as their number is given by the rank of the group, $n=rank[G]$, and $rank[SU(5)] = 4$. An interesting property of GUTs is, for example, that they predict that the proton can decay (which has not been observed to this day, as its lifetime is calculated to exceed $10^{29}$ years, as it is ``the lightest baryon, has nowhere to go; conservation of baryon number guarantees its absolute stability'' \cite{Griffiths}), mediated by a gauge boson with a mass of order $\order{E_{GUT}}$. The $E_{GUT}$ is also very important, as it is ``the energy where the three running gauge coupling constants of the SM gauge group become equal'' \cite{Leike}, which is a very important aspect of some GUTs, since they try to also unify the coupling constants of the SM. Only precision measurements at LEP and SLC could prove that they do not meet at the point predicted by the $SU(5)$ GUT \cite{Griffiths}\cite{Leike}.

If this does not happen, and continuing with the study of GUTs, one extra neutral gauge boson, $Z^{\prime}$, is (at least) predicted by gauge groups larger than $SU(5)$. As was shown by H. Fritzsch and P. Minkowski, the next group of interest larger than $SU(5)$ is $SO(10)$, predicting exactly one extra neutral gauge boson ($rank[SO(10)]=5$) $Z^{\prime}$ \cite{Leike}. Because of this, even though many of these new physical models are very different in terms of the underlying theory, many of them predict the existence of this new $Z^{\prime}$ boson at different mass scales, but more importantly for current research, at the TeV scale. 

For this reason, some of the biggest particle experiments in the world such as ATLAS and CMS at the LHC have ``an extensive physics program to search for $Z^{\prime}$'' \cite{Florez}, since a ``new heavy vector boson would likely be one of the first clearly visible signals for new physics to be detected by an experiment'' \cite{Hayden}, being an important part of the scientific program of present and future colliders \cite{Leike}.

Some of the theories that predict a $Z^{\prime}$ boson are for example the Sequential Standard Model (SSM), where it has the same couplings as the $Z^0$ from the SM but with a higher mass. The easiest way to do this is by incorporation of a new $U(1)$ sector to the SM, as in the $E_6$ GUT, which for instance adds two new $U(1)$ groups through the decomposition $E_6 \rightarrow SO(10)\times U(1)_{\psi} \rightarrow SU(5)\times U(1)_{\chi}\times U(1)_{\psi}$, where the $SU(5)$ part contains the SM. Because of this, backgrounds are always present as the same processes that produce the $Z^0$ and $\gamma$ from the SM, produce the $Z^{\prime}$ \cite{Leike}\cite{Hayden} (see for example Figure \ref{z'}).

\begin{figure}[ht]
    \centering
    \includegraphics[width=7cm]{Sections/images/Imagenes/z'.png}
    \vspace{-1\baselineskip}
    \caption{Example of the leading-order Feynman diagram for production and decay of Z’ into final states involving b quarks. Taken from \cite{Aaboud}.}
    \label{z'}
\end{figure}

Other very interesting GUT is the B-L model, which also comes from the $SO(10)$ GUT, where a new $U(1)$ group is added, including another complex scalar Higgs field ($\chi$, a singlet, whereas $\Phi$ from the SM is a doublet) is required for the spontaneous symmetry breaking of the B-L symmetry \cite{Balasubramaniam}. Other type of B-L model is the Left-Right Symmetric Model (LRSM), where a right-handed group is added to the electroweak sector of the SM, at high energies replacing $SU(2)_L \rightarrow SU(2)_L \times SU(2)_R$, and $U(1)_Y \rightarrow U(1)_{B-L}$, restoring parity, by generating new bosons $W'^{\pm}$ and $Z^{\prime}$ coming from the $SU(2)_R\times U(1)_{B-L}$ sector (just as the electroweak sector was generated in the SM by the $SU(2)_L\times U(1)_Y$ symmetry), where the $W^{\prime}$ could couple to right-handed neutrinos \cite{Hayden}.

One last type of theories that predicts the existence of $Z^{\prime}$ bosons emerges in universal extra dimensions (UED) theories, where the forces are unified in higher dimensions, in the same way as Kaluza showed gravity and electromagnetism can be unified in a five dimensional space \cite{Kaluza}. Kaluza-Klein theories are extensions of the SM, in which a fifth dimension is considered, where our four-dimensional world is embedded in the fifth-dimensional space called the bulk, where we can only observe four dimensions (as the fifth dimension is compactified). In these theories masses can be measured as an excitation of higher dimensions, generating a resonance at high energies that can be measured \cite{Huber}\cite{Riemann}.

\subsection{$Z^{\prime}$ model} \label{ssec:zpmodel}

The $Z^{\prime}$ is a new massive vector boson predicted in different theories by a new $U(1)^{\prime}$ group symmetry in the SM. In some theories, due to the $U(1)^{\prime}$ symmetry breaking, and similar to the $U(1)_Y\times SU(2)_L$ symmetry breaking, a new doublet (or singlet in some theories) higgs bosons emerges, giving the $Z^{\prime}$ its mass. Some of these theories predicting new higgs bosons also allow for a mechanism with which neutrinos can acquire their mass \cite{Leike}.

Since this new boson is a massive vector boson, it is described by a Proca-like mediator, such as in Equation (\ref{propproca}). Then, the propagator of the new $Z^{\prime}$ boson is given by
\begin{equation*}
    iD_{\mu\nu} = i\dfrac{-g_{\mu\nu} + k_{\mu}k_{\nu}/m_{Z^{\prime}}^2}{k^2 - m_{Z^{\prime}}^2}.
\end{equation*}

In this last expression, $k$ is the momentum of the $Z^{\prime}$ and $m_{Z^{\prime}}$ its mass. Then, for $Z^{\prime}$ masses at the scale of TeV, it is possible to drop the term related to $k_{\mu}$ and $k_{\nu}$, simplifying the propagator. With this propagator one could attempt on computing the Feynman amplitude and cross sections of different processes involving this boson, nevertheless, and for this thesis, the processes studied present a very complicated topology involving several intermediate and final state particles, which makes calculations very difficult. Because of this reason, feynRules is used: a mathematica-based package used to numerically generate the cross sections of different events, given a theoretical model.

Finally, since there are no quantum numbers which forbid mixing between the $Z^{\prime}$ and the $Z$ \cite{Leike}, some theories predict a mixing between these two gauge bosons. Therefore, with the mass matrix obtained, with a diagonal term $\delta M$, it is easy to obtain the theoretical mass the new $Z^{\prime}$ boson will have, which has a value of

\begin{equation*}
    M_{1,2}^2 = \dfrac{1}{2}\left[M_Z^2 + M_{Z^{\prime}}^2 \pm \sqrt{(M_Z^2 - M_{Z^{\prime}})^2 + 4(\delta M^2)^2}\right].
\end{equation*}

\subsection{Searches for the $Z^{\prime}$} \label{ssec:searcheszp}

Some of the latest results obtained on the actual limits of the energy scale at which searches of $Z^{\prime}$ bosons are conducted in the ATLAS and CMS detectors, are shown in Figures \ref{resultsATLAS} and \ref{resultsCMS} respectively for Drell-Yan processes. There, the green and yellow sections are the fringes of maximum likelihood, which represent one and two standard deviations away from the expected value of the cross section (dark dashed line), where the observed values (black dotted line) being inside the five deviations limit shows no sign of new physics below the established limits, which vary depending on the specific model. 

These limits are set as this is the point where the experimental data and the theoretical cross section (red and/or blue dashed lines, depending on the model) intersect, where the experimental data being below the theoretical line means that the experiment has enough sensitivity to exclude this region, as no signs of new physics were found, with a certainty of 95\% confidence level, using a maximum likelihood test statistic approach. Above this point, the experimental sensitivity is not sufficient to rule out the existence of new physics, and so, above this point it is still possible that the $Z^{\prime}$ could be found.

Previous studies such as \cite{Aaboud} and \cite{Sirunyan} had set the limit of the $Z^{\prime}$ mass at 2.1 TeV, but more recent references such as \cite{results1} and \cite{results2} have set the limits, for light leptons (electrons and muons) final states (if they are not found at the LHC) of $Z^{\prime}$ for some specific models at $Z^{\prime}_{\psi}=5.8$ TeV, $Z^{\prime}_{\textrm{SSM}}=6.4$ TeV for a center of mass energy of $\sqrt{s}=14$ TeV, and $Z^{\prime}_{\textrm{TC2}}$ from 3.9 up to 4.7 TeV for decay widths of 1\% and 3\%, for a center of mass energy of $\sqrt{s}=13$ TeV. The subindex $\psi$ refers to the study of the $Z^{\prime}$ is the E6 GUT, SSM to the Sequential Standard Model and TC2 to the topcolor-assisted-technicolor model. The graphs of some of the results reported on both papers are shown below.

\begin{figure}[!htb]
     \begin{center}
        \subfigure[Latest simulations for the expected sensitivity searches of $Z^{\prime}$ at ATLAS, for the E6 model, in the $Z^{\prime} \rightarrow ll$ channel]{
            \label{resultsATLAS}
            \includegraphics[width=0.4\textwidth]{Sections/images/Imagenes/ATLAS1.png}
        }
        \subfigure[Latest simulations for the expected sensitivity searches of $Z^{\prime}$ at ATLAS for the TC2 model.]{
           \label{resultsCMS}
           \includegraphics[width=0.42\textwidth]{Sections/images/Imagenes/ATLAS2.png}
        }
    \end{center}
    \vspace{-1\baselineskip}
    \caption{Graphs of the results of some of the latest searches for $Z^{\prime}$ bosons at scales of TeV in ATLAS for different models. Taken from \cite{results1} and \cite{results2}.}
   \label{limits}
\end{figure}

%HABLAR MAS EN ESTE CAPITULO PARA HACER UNA MEJOR TRANSICION ENTRE EL CAP 2 Y EL 4/5: MATRIZ DE MIXING, VALORES DE M\_Z, PROPAGADOR PARA EL Z', DIAGRAMA Y APLITUD DE FEYNMAN PARA EL PROCESO (MENCIONAR QUE ES DIFICIL DE CALCULAR POR LO COMPLEJA QUE ES Y LA SUMA SOBRE LAS 8 POSIBLES POLARIZACIONES DE GLUONES, MOTIVAR EL FEYNRULES)